\chapter{Resultados} 
\label{cap:results}

En este capítulo, se procederá a mostrar los resultados obtenidos al integrar el algoritmo propuesto en los casos de usos descritos. 

Aunque el proyecto \ac{RASimAs} ha impuesto una serie de requisitos para generar la herramienta \ac{TPTVPH}, se diseñó con la idea de ser lo más flexible y permitir el desarrollo de los módulos por separado, solo haciendo enfasís en los formatos de intercambio entre las distintas etapas. Esta herramienta permitiría crear una base de datos de modelos anatómicos que se podrían utilizar en el simulador \ac{RASim}. Este simulador ha sido desarrollado por varios participantes del proyecto, siendo aglutinados en última instancia por el \ac{Courseware}. Esta aplicación es la encargada de supervisar la interacción del usuario con el sistema y proporcionar un entorno de aprendizaje del procedimiento de \ac{RA}.

En el caso de uso del simulador de diagnóstico radiológico, la integración del algoritmo de posicionamiento puede demostrar su flexibilidad y versatilidad. Este podría ser integrado en cualquier entorno que necesitara incorporar un método que permita la animación de modelos anatómicos con estructuras internas.
\todo{mejorar}


