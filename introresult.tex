\chapter*{Resultados} 
\label{cap:results}

En este capítulo, se procederá a mostrar los resultados obtenidos tanto con el algoritmo propuesto como de su integración en los caso de usos descritos. 
%
% El objetivo es mostrar las dos características principales: 
% \begin{itemize}
%     \item El algoritmo puede deformar modelos anatómicos con estructuras internas aunque estos se encuentren incompletos o no se dispongan de sus propiedades mecánicas. 
%     \item El algoritmo puede ser integrado en cualquier entorno que necesitara incorporar un método que permita la animación interactivamente de modelos anatómicos con estructuras internas.
% \end{itemize}
% Unelo a parrafo anterior
En primer lugar, se analizarán los resultados obtenidos del algoritmo de posicionamiento de pacientes virtuales. Se detallarán los experimentos realizados para evaluar los resultados utilizando distintos modelos y el rendimiento.
%Se mostrarán los datos respecto a su rendimiento y calidad visual. 
%Además, se mostrará la opinión de usuarios de los resultados obtenidos en la fase de optimización. 
%Por último, se discutirán los resultados en la sección \ref{posing:discusion}.

En segundo lugar, se hablará de la integración del algoritmo dentro del proyecto \ac{RASimAs}. La herramienta \ac{TPTVPH} proporcionaba la capacidad de adaptar la postura del paciente virtual en la \emph{suite} \ac{ITGVPH}. %crear una base de datos de modelos anatómicos para el simulador \ac{RASim}. 
Después, se describirán los resultados del prototipo \ac{RASim}. Primero, se describirán los resultados de cada componente que conforma el prototipo y, entre ellos, el módulo \acs{Courseware}, encargado de supervisar la interacción del usuario con el simulador y proporcionar un entorno de aprendizaje del procedimiento de \ac{RA}. Finalmente, se discutirán los problemas encontrados para realizar el ensayo clínico para la validación del prototipo. %Finalmente, se discutirán los resultados en la sección \ref{rasim:discusion}.

Por último, se mostrarán los resultados del simulador de radiología diagnóstica. En esta aplicación, el algoritmo de posicionamiento demuestra su flexibilidad y versatilidad, permitiendo al usuario modificar la postura del paciente virtual interactivamente, mientras se puede observar la imagen de rayos X en tiempo real. Se mostrarán los resultados obtenidos y se detallará los experimentos realizados para la validación de la herramienta. %los datos respecto a su rendimiento y calidad %. 
%Por último,  se discutirán los resultados en la sección \ref{xray:discusion}.


